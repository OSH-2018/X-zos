\chapter{unikernel技术的思想}

\section{容器化技术解决的问题和实现原理}

容器化技术,这里以docker为例,给云计算平台上的应用程序开发带来了一场革命,docker技术解决的问题就是软件的运行环境。想法非常
简单:把软件和软件的依赖环境(运行库等)打包成一个整体,这样开发者不需要再担心开发测试环境和云平台环境的差别了。

DOCKER的成功是无疑的,它采用了类似虚拟化的方式。容器之间需要共享一个操作系统,也分享操作系统提供的服务,这与传统的虚拟机不同
。传统的虚拟机真的模拟出一个与宿主机独立的环境,拥有自己的软件和库,容器化技术相当于解决了不同虚拟机各自拥有了一套独立而重复的
运行环境的问题。

除了更少的空间占用外,相比传统虚拟机,docker运行的效率更高,这是因为它和其他的进程一样运行,而不用在一个虚拟化出来的相对较慢
的计算机上运行。并且它也不需要像传统虚拟机一样等待系统boot,它就在host系统上运行。

相比传统虚拟机,docker在资源占用,运行速度和效率上都有优势,但是它并没有解决安全问题,这似乎并不是docker的设计初衷。

\section{Unikernel解决的问题和实现原理}

相比以docker为代表的容器技术,Unikernel是更好的选择,它确实比docker做得更加极端:不仅打包了程序运行所依赖的库,还有它需要
的系统服务,也就是说,它完全作为一个专门化的虚拟机镜像运行,而不需要任何多余的模块。

Unikernel 为了实现体量的缩小,选择了与Docker完全不同的路线。它只专注于一个目标程序,没有其它冗余的程序和支持,没有多进程切
换,所以系统很小也很简单。虽然 Docker 容器的 image 比传统的虚拟机(以G计)要小很多,但是一般也是好几百兆,而 unikernel 由
于不包含其它不必要的程序和服务(ls,echo,cd,tar等),甚至没有Shell,所以体积非常小,通常只有几兆甚至可以更小。比如 mirageOS
的示例 mirage-skeleton 编译出来的 xen 虚拟机只有 2M。体量的缩小也使得Unikernel的启动和运行的速度相较 Docker 有明显的提高。

另外,Unikernel 没有其它不必要的程序,使其受到漏洞攻击的可能性大大降低,docker 里面虽然大多数情况只跑一个程序,但是里面含有
其它 coreutils,只要其中任何一个有安全漏洞,整个服务可能就会受到攻击。而Unikernel 中没有 Shell 可用,没有密码文件,没有多余
的设备驱动,只有少数功能可以利用,这严重限制了未经授权的用户可能造成的破坏。另一方面,对于Unikernel而言,每一个操作系统都是定制用途的,其中包含的核心库均不相同,即使某个服务构建的操作系统由于特定软件问题而遭到入侵,同样的入侵手段在其他的服务中往往不能生效。这无形中增加了攻击者利用系统漏洞的成本。

这样的设计依赖于“库操作系统”,这并不是一个真正的操作系统,Unikernel的开发过程与嵌入式系统很像,类似于”交叉编译“,对于编好的
程序,我们需要找到它运行所需要的最少条件,从库操作系统中选择出来与程序打包成虚拟机镜像。这意味着它很可能不能运行其他的程序。

Unikernel System,一个知名 Unikernel 开发研究组织在2016年一月被Docker收购。作为当下最流行的应用容器引擎服务的提供商,Docker 
这一收购足以说明 Docker 对于 Unikernel 的重视。

Unikernel的实现:在内存中,一个传统的应用程序是这样的:
(picture of normal application stack)

软件可以被分为内核空间和用户空间,内核空间有操作系统提供的服务和运行库,包括一些底层的硬件驱动,文件系统,内存管理等等。这样的设计
也同时提供了用户与系统的隔离,进程的调度,以及其他多用户操作系统需要的功能。用户空间则提供具体的应用程序功能,在计算机的使用者看来
,用户空间包含的是要运行的指令,而内核空间为用户空间的代码提供具体的功能实现的服务。

然而在Unikernel中,软件栈变得不一样了:

(picture of Unikernel application stack)

可以看到这里已经没有用户空间和内核空间之分了,传统的设计是内核,运行库,应用程序。而在Unikernel中所有都是一体的,镜像中也只有一
个应用程序运行,这里的代码已经包含了完成工作需要的所有代码。只需要作为一个系统镜像启动,完成所有的功能。

看起来这种设计理念貌似是非常原始的,然而从操作系统的设计上出发我们会发现,最早人们设计系统调用,就是为了节省程序员在编写底层驱动上
消耗的时间,专注于软件功能本身。实现这个目的的方式不仅有是设计一个“包罗万象”的操作系统,我们还可以根据运行程序的需要,为每个应用
程序“定制”操作系统,当然这个定制过程是自动化的,过程很像交叉编译。

如果我需要运行很多不同的操作系统呢?很简单,只需要为每个应用程序都制作Unikernel,再开多个虚拟机就可以了。

\section{Unikernel与docker的比较}

Unikernel,Docker 二者经常被拿来相提并论。它们的主要意义都是用来替换云计算中使用传统虚拟机这一过于臃肿和有时不太安全的解决方案,
但两者的区别非常明显。

Docker 为云计算的操作系统提供了相较传统虚拟机一个更轻量级的选择。在很多方面,Docker 下的容器与虚拟机非常相似。它们都旨在为运行代
码提供一个能支持代码运行的独立稳定的环境。最大的区别是 Docker 通过分享来减少重复。Docker 可以共享主机环境的 Linux 内核,也可以
共享操作系统的其余部分,甚至它们还可以共享除应用程序代码和数据之外的所有内容。例如,我可以使用容器在同一台物理机器上运行两个博客。
这两个容器可以设置为共享除模板文件,媒体上传和数据库之外的所有内容。借助一些复杂的文件系统技巧,每个容器都可以“认为”它具有专用的文
件系统。
