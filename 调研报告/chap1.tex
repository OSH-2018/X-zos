\chapter{云计算领域面临的挑战}
\thispagestyle{empty}

\section{引言}
云计算的出现是互联网的传输速度,计算机计算能力的增强和个人、企业等计算机用户对计算资源的需求增加的必然结果,用户的数据,
应用程序等在云计算中心托管,大大节省了用户自己的维护成本,然而这也给云计算平台提出了诸多挑战:
\begin{itemize}
    \item 隔离:一方面是用户之间的隔离,另一方面是用户与操作系统的隔离,保证用户不会造成破坏。
    \item 性能:性能意味着托管的成本,高性能的云计算平台自然在一定的开销下能给用户提供更优质的服务。
    \item 安全:用户的数据,系统的运行状态等一旦被骇客入侵,将造成极大的损失。
\end{itemize}

\section{隔离}
在个人用户的操作系统中,隔离已经十分重要,用户可能会运行很多应用程序,应用程序之间必须隔离,做到“一个应用程序的崩溃不能影
响其他应用程序,更不能影响操作系统。”因此我们必须引入某种隔离机制,保证应用程序之间,应用程序和系统的隔离。在内存上,我们
引入了用户空间和内核空间的分离,在CPU的运行上我们设置了不同的运行级别,一些敏感的操作被设定在更高级别的运行状态下,这样能
阻止本不应该运行的指令破坏系统。

而在云计算平台上,这样的隔离变得更加复杂——体现在虚拟机上,我们无法保证用户托管的应用是安全的,也无法保证用户做出安全的操作
,因此我们引入更强的隔离机制——虚拟机。用半虚拟化或者完全虚拟化的方式模拟一台物理机器,用户的应用程序和操作在上面的运行同一台
真实机器上的效果完全等价,这样用户可以随心所欲地在上面执行各种操作而不会破坏宿主操作系统,在理想情况下(没有严重漏洞)也不能
访问其他的虚拟机空间。

\section{性能}
虚拟化技术引入的隔离机制很好地保证了整个体系的运行和管理,但是无疑也加重了云计算平台的负担。这里需要说明软件栈的概念——指一个
应用程序运行所依赖的环境,包括运行库,操作系统中的硬件驱动,进程管理,内存管理等模块。在传统的虚拟机模式下,我们的宿主机已经
有了一个相当庞大的软件栈,而我们又在这基础上搭建虚拟化的操作系统,又一次增高了软件栈。这样的解决方案无疑是低效的。

这样的状况在云计算中心意味着什么呢?我们需要浪费大量的空间去执行非常简单的任务,执行这些任务之前的初始化占用的资源甚至比应用程
序本身还要多。

为了改进性能,一种以Docker为代表的应用程序级的虚拟化技术迅速发展起来,docker其实是一种将应用程序和运行库,软件依赖打包分发
的技术,再结合镜像站等网络服务,为开发者提供了一个非常完整的开发链,因而迅速得到了普及。

还是由于软件栈的原因,每个应用程序都有它的运行环境,这些环境是由开发者使用的技术决定的,然而这并不是好事,因为开发者使用的环
境和真实的运行环境很难做到完全一致。比如开发者编写的程序依赖的库,库的版本,操作系统等等都会和使用场景不一样,这样给开发和测
试造成了很大的不方便。

综合上述面临的问题来看,很多问题都涉及到同一个点:操作系统太过复杂,使我们很难确保执行每一个功能的模块都是安全的,还会让操作
系统的效率降低,还会使开发与部署变得非常繁琐。然而在云计算平台上,这样的缺点有望通过容器技术得到解决。

\section{安全性}
无论是个人PC机,云计算平台还是嵌入式系统,安全性从来就没有一个一劳永逸的解决方案,也许这不仅是一个技术问题,也是一个哲学问题。
但无论如何,现在操作系统的复杂程度决定了我们很难真正掌握一个系统的安全状况。最常见的模式是:发现一个漏洞,修复一个,以后再发
现漏洞。现有操作系统的冗余复杂,使得对计算机系统的维护变的十分困难,并且留下了大量的漏洞和攻击面(attack surface),给恶意黑
客以可乘之机。类似的例子有很多,拿windows10为例,仅2016年一年内,Microsoft为其修补的漏洞就达729个之多。“WannaCry”病毒
也是利用windows系统中的“Enternalblue”漏洞来进行进攻的。

尽管安全问题可以通过内核的可信化验证或者使用新的编程语言--比如RUST,得到解决,但是我们必须承认,功能越是复杂的系统,安全越成问题。

安全问题在云计算时代直接与大量的宝贵信息和计算资源挂钩,云计算中心最禁不起严重漏洞造成的风险。Amazon, Microsoft, Yahoo,
Linkin等公司都遭受过攻击,造成用户的信息泄漏,严重威胁用户的安全。2017年5月,“WannaCry”勒索病毒席卷全球,数十万电脑用户
中招,造成的直接经济损失超过80亿美元。同年10月,东欧银行网络遭到黑客入侵,损失达上亿元。据不完全统计,每年全球仅因黑客攻击导
致的直接经济损失达4500亿美元。从个人计算机到银行、政府的大型主机,无一不面临着黑客、木马等不安全因素的威胁。