\chapter{Unikernel的生态}

一项技术的广泛使用需要开发者为其搭建方便开发、部署、管理的生态环境。目前为止,unikernel的开发者的项目有:

\section{Jitsu}

Jitsu 是 “Just-in-time summoning of unikernels”的缩写,事实上它是一个DNS服务器,当它收到请求时,会快速启动Unikernel来对
请求者提供服务,这样相当于快速地生成了服务器,简单来说,它就是一个即时启动Unikernel的服务器。

Jitsu的作用就在于它实现了一个叫临时微服务器的概念(Transistent microserver),因为Unikernel的boot时间只有几十毫秒,是很理想的
动态建立服务器的方式。

\section{Rump Kernels}

Rump Kernels的设计理念基于NetBSD,一个跨平台性能很好的操作系统,为了实现很好的跨平台性,NetBSD的设计非常适用于各种不同的硬件平台,
替换某个硬件驱动后仍可以正常运转,所以NetBSD可以在很多硬件平台上部署。

Rump Kernels是Rumprun(一个支持POSIX的Unikernel实现)的基础,因为在Unikernel的开发中,需要做的其实是一个交叉编译,目标平台
与自己的机器并不相同。

\section{Xen Project Hypervisor}

Unikernel的本质是一台虚拟机,而Xen虚拟化技术提供了一种叫paravirtualization的技术。这项技术使得一些虚拟机“知道”自己在一个H
ypervisor上而不是一台真正的机器上。那么虚拟机与Hypervisor的交互就不需要“模拟”一台真实的机器了。

这项技术使得Xen项目成为Unikernel的首选运行平台,paravirtualization使得Unikernel可以非常高效地与宿主机进行通信,使得启动并
管理大量的Unikernel成为可能,目前,在运算规模随Unikernel的数目成线性增长的情况下,hypervisor可以同时运行大约1000个Unikernel。

\section{Clive OS}

Clive 是一个云操作系统,旨在提供简单高效的云计算服务,目前主要被用来虚拟机上提供进一步的开发支持。

简单来说,Clive是一个用Go语言设计的Unikernel实现,设计者修改了Go语言的编译器和Runtime来提供更好的网络服务。

\section{MiniOS}

MiniOS 是在Xen项目的基础上的操作系统,MiniOS可以用来运行大量的Unikernel应用,包括MiregeOS和ClickOS。而对于自己来说,MiniOS什
么也不做,它的价值在于很容易被修改使Unikernel可在上面运行。它简化了Unikernel的开发周期。